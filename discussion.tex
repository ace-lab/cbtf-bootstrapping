\section{Discussion}

\subsection{Why Not Canvas?}

\subsection{Exam Lengths}

Our exams were heavily based on existing quizzes and final exams for
the course, hence the 3 hour time limit for the final.
We have since moved to a 2 hour final (really 1 hour and 50 minutes,
since each CBTF slot

\subsection{Policy}

On our campus, a course is classified as either having a ``written final
exam'' or not.  If it does, it is assigned a specific time and room for
the final exam during finals week.

Can your students take the final exam during a time other than the assigned time during
finals week, or even during a different week?  (In our case, we
ensured that some of the available slots did coincide with the
assigned finals
week slot, and offered to accommodate any students who ended up
being unable to use any other slot due to scheduling conflicts. We had
no such students.)
Is a CBTF exam ``written'' even though it's administered on a
computer?  
These administrative minutiae, along with determining the number of angels that can
dance on the head of a pin, we leave to discussions between you and
your central campus administration.  We suggest clearing your plan
with your students, and asking forgiveness
rather than permission if you end up accidentally transgressing the
letter of the law.  Our experience was that the vast majority of
students greatly appreciated the flexibility of being able to schedule
their final exam time.\footnote{Indeed, in a follow-up survey with a much larger
($n=~750$) upper-division course that used the CBTF in Summer 2025,
this flexibility was the most frequently cited benefit among students
who preferred the CBTF to traditional paper based exams.  48\%
definitely preferred the CBTF, 30\% definitely preferred paper, and
22\% had no strong preference.}

Further, as Figure~\ref{fig:score_histogram}
shows, even though our final included only a modest amount of
randomness, there is no obvious evidence of cheating by ``information
leakage'' from students who took the exam earlier to those who took it
later in the 4-day window.

\begin{figure}
  \caption{\label{fig:score_histogram}
        The final exam was administered over a four-day window, with
        three 3-hour sessions available each day, to a total of 217
        students. The overall statistics $\mu=\mbox{median}=68, \sigma=12$
        track the daily statistics.   (TO DO: run a chi-squared test
        or something on this.)}
  \includegraphics[width=\textwidth]{figs/score_histogram}
\end{figure}



