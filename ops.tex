\section{Operations}

\textbf{Lesson:} Do a \emph{comprehensive\/} dry run

\subsection{Scheduling}

Our final exam was based heavily on the previously-administered
pen-and-paper final, and consisted of some ``simple'' questions
(selected-response, short answers, etc.) and some bespoke questions
that we manually coded in PrairieLearn.  The original final was
designed to be completed just over 2 hours, and our policy had always
been to allow 3 hours so that students wouldn't lose points due to
running out of time.\footnote{These timings are baseline, not
considering extra-time accommodations, which we discuss separately.}

Since we weren't sharing the facility with any other courses, we
scheduled the room in 3-hour ``shifts'' and had students sign up for
seats during one particular shift over a 4-day window during finals week.

we did a 3 hour final. it's down to 2 hours now.  We therefore did 3-hour

\subsection{Reservations}

If not using PT, use SignupGenius or similar for reservations. But DSP will be tricky.

Introduce friction for no-shows.

Have 10-15\% spare capacity (~2 machines in a room of 30)



\subsection{Proctors}

We used our TAs and course staff as proctors for the pilot.  (One
proctor is enough for the small room we had.)  But it's important that the
processes they use are the same as those you'd use at scale, as much
as possible.
Key among these is \emph{not answering clarification questions} that
students have on the exam, since (especially so with an asynchronous
exam) it's impossible to be fair about who hears which clarifications.
Instead, instruct students to make a note on the exam (PrairieLearn
has a specific feature for this), and if you have time between exam
shifts, repair the question, and if necessary, apply manual grading
changes to the cohort of students that experienced the bug.

Make sure proctors know how to access the room, and have practiced the
precise setup/teardown procedures to leave it as you found it.

We did not have any ``tech support'' for PrairieLearn or PrairieTest
outside the small team of people involved in the research pilot and
the Slack channel for PrairieLearn (which is extremely helpful, but
much more helpful if you're running hosted PrairieLearn, meaning their
staff can see your content and configuration if necessary to help you
troubleshoot problems).
 
\subsection{Preparation/Messaging to Students}

Give crystal clear instructions on both the login to the machine process (if applicable) and logging into the exam. We had slips of paper that we handed out as people entered the room.

A key tenet of UIUC's CBTF is that no paper enters or leaves the room,
ever.
See the discussion on Cheat Sheets below, but scratch paper
color-coded to each session (to thwart infiltrating/exfiltrating data)
is distributed by proctors to anyone who requests it, and is collected
at the end of the session.

For a while after we began hosting multiple courses at the same time
in the CBTF, we allowed instructors to specify that students could
hand in scratch paper to the proctors, and a course TA would come by
later to collect it, ostensibly to enable partial credit for
long-answer questions.  This created extra work for proctors and
course staff, required a safe place to store these papers after an
exam without getting them mixed up between courses, and led to
problems such as students not labeling scratch paper with their names,
scratch paper getting lost in transit when the student swore they had
given it to the proctor, and so on.  We soon
abandoned the practice, and today PrairieLearn has an image-capture facility
that (assuming an inexpensive Web cam is connected to each testing
station) allows scanning and uploading written scratch notes as part
of an exam question.  We don't plan to reinstate the ``scratch paper
can be turned in'' policy.

In a similar vein, we initially allowed cheat sheets.  Don't do
it. Even for our own class, proctors couldn't tell if the cheatsheet
was ``legal'' (was the student allowed one or both sides?  Did it have
to be written out by hand or were machine-printed sheets acceptable?
If the former, was there an exception for students with a disability
accommodation related to handwriting?  The list goes on and on.)  It
was obvious that with students in more than one course taking exams
simultaneously, as would occur in steady state, any kind of cheat
sheet would be untenable, so we now have a blanket ``no paper cheat sheets,
no exceptions'' rule.  Fortunately, PrairieLearn makes it possible to
both embed a common PDF or HTML cheat sheet in the exam itself, and
(less obviously) a student-created cheat sheet that has been
explicitly approved by the instructor and can be made available to
that particular student during the exam.


Make sure students know where the room is! It seems obvious but many
of our students had never had occasion to use one of the shared labs,
as it was in a building that rarely hosts computer science courses.

Make sure building access is not an issue.  In one case we had to post
a proctor at the building door because the building required cardkey
entry after 6PM, and most students did not have cardkey access.
(Getting cardkey access to a room for a student is a bureaucratic
procedure that doesn't happen overnight; in our case, it's fine for a
few proctors, but not for hundreds of students enrolled in a course.)

\subsection{Messaging to Faculty}

For faculty considering CBTF adoption, the biggest stumbling block has
been accepting the fact that \emph{not all of their students will take
the exam at the same time.}

Faculty initially thought they could ``block out'' the CBTF just for
themselves, or that at least they could block out consecutive sessions
in a day to minimize the likelihood of information leakage.  With one
or two courses, you can do this (at great cost in pain, since
invariably students get sick, forget, no-show, etc., requiring them to
then make another reservation outside of their course's designated
``block''), but with more courses, it torpedoes anything resembling
high-utilization scheduling.  Furthermore, accommodating faculty in
this way simply encourages them to continue developing exams the old
way, in which knowledge of specific questions becomes an advantage.

Our consistent message to faculty now is that they cannot do this,
period.  Even if their course enrollment is smaller than the CBTF
capacity, we cannot block it for them.  Finding a ``coalition of the
willing'' who could deal with this restriction allowed us to grow to
the point where demand now exceeds supply, so we can comfortably
enforce this restriction.
\cite{fox-cbtf-not-proctoring} explains other reasons why it is
important to message to faculty that the CBTF is \emph{not} a general
proctoring facility that allows them to appraoch exam authoring and
administration as they have always done but just outsource the
proctoring; rather, effective use of the CBTF does require some
workflow changes, but they amortize very well and help is available to
make the transition.  For example, one of our recent courses had a
large bank of perhaps 400 questions for use on midterms and exams, and
are happy with each student getting a random subset (with some
constraints) as their exam.
