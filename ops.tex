\section{Operations}

\textbf{Lesson:} Do a \emph{comprehensive\/} dry run

\subsection{Scheduling}

Our final exam was based heavily on the previously-administered
pen-and-paper final, and consisted of some ``simple'' questions
(selected-response, short answers, etc.) and some bespoke questions
that we manually coded in PrairieLearn.  The original final was
designed to be completed just over 2 hours, and our policy had always
been to allow 3 hours so that students wouldn't lose points due to
running out of time.\footnote{These timings are baseline, not
considering extra-time accommodations, which we discuss separately.}

Since we weren't sharing the facility with any other courses, we
scheduled the room in 3-hour ``shifts'' and had students sign up for
seats during one particular shift over a 4-day window during finals week.

we did a 3 hour final. it's down to 2 hours now.  We therefore did 3-hour

\subsection{Reservations}

If not using PT, use SignupGenius or similar for reservations. But DSP will be tricky.

Introduce friction for no-shows.

Have 10-15\% spare capacity (~2 machines in a room of 30)



\subsection{Proctors}

Use your TAs at first. It's a pilot. BUT have them act like it's the real thing: don't answer questions.

1 proctor is enough for small rooms

Make sure proctors know how to access room, and precise setup/teardown procedures to leave it as you found it

proctors vs tech support vs front-desk

scratch paper

 
\subsection{Preparation/Messaging to Students}

Give crystal clear instructions on both the login to the machine process (if applicable) and logging into the exam. We had slips of paper that we handed out as people entered the room.

A key tenet of UIUC's CBTF is that no paper enters or leaves the room,
ever.
See the discussion on Cheat Sheets below, but scratch paper
color-coded to each session (to thwart infiltrating/exfiltrating data)
is distributed by proctors to anyone who requests it, and is collected
at the end of the session.

For a while after we began hosting multiple courses at the same time
in the CBTF, we allowed instructors to specify that students could
hand in scratch paper to the proctors, and a course TA would come by
later to collect it, ostensibly to enable partial credit for
long-answer questions.  This created extra work for proctors and
course staff, required a safe place to store these papers after an
exam without getting them mixed up between courses, and led to
problems such as students not labeling scratch paper with their names,
scratch paper getting lost in transit when the student swore they had
given it to the proctor, and so on.  We soon
abandoned the practice, and today PrairieLearn has an image-capture facility
that (assuming an inexpensive Web cam is connected to each testing
station) allows scanning and uploading written scratch notes as part
of an exam question.  We don't plan to reinstate the ``scratch paper
can be turned in'' policy.

In a similar vein, we initially allowed cheat sheets.  Don't do
it. Even for our own class, proctors couldn't tell if the cheatsheet
was ``legal'' (was the student allowed one or both sides?  Did it have
to be written out by hand or were machine-printed sheets acceptable?
If the former, was there an exception for students with a disability
accommodation related to handwriting?  The list goes on and on.)  It
was obvious that with students in more than one course taking exams
simultaneously, as 


NO CHEATSHEETS. Headache to validate, won't scale.

Make sure students know where the room is! Seems obvious but...

Make sure building access is not an issue (bldg hours vs CBTF hours). We had to post a proctor at building door because of this.

\subsection{Messaging to Faculty}

can't block it out

can't rely on sync exams. period. even in a small class. cite my 'not
proctoring' paper.
