\section{Operations}

\textbf{Lesson:} Do a \emph{comprehensive\/} dry run

\subsection{Scheduling}

Our final exam was based heavily on the previously-administered
pen-and-paper final, and consisted of some ``simple'' questions
(selected-response, short answers, etc.) and some bespoke questions
that we manually coded in PrairieLearn.  The original final was
designed to be completed just over 2 hours, and our policy had always
been to allow 3 hours so that students wouldn't lose points due to
running out of time.\footnote{These timings are baseline, not
considering extra-time accommodations, which we discuss separately.}

Since we weren't sharing the facility with any other courses, we
scheduled the room in 3-hour ``shifts'' and had students sign up for
seats during one particular shift over a 4-day window during finals week.

we did a 3 hour final. it's down to 2 hours now.  We therefore did 3-hour

\subsection{Reservations}

If not using PT, use SignupGenius or similar for reservations. But DSP will be tricky.

Introduce friction for no-shows.

Have 10-15\% spare capacity (~2 machines in a room of 30)



\subsection{Proctors}

Use your TAs at first. It's a pilot. BUT have them act like it's the real thing: don't answer questions.

1 proctor is enough for small rooms

Make sure proctors know how to access room, and precise setup/teardown procedures to leave it as you found it


\subsection{Preparation/Messaging to Students}

Give crystal clear instructions on both the login to the machine process (if applicable) and logging into the exam. We had slips of paper that we handed out as people entered the room.

NO CHEATSHEETS. Headache to validate, won't scale.

Make sure students know where the room is! Seems obvious but...

Make sure building access is not an issue (bldg hours vs CBTF hours). We had to post a proctor at building door because of this.

