\section{Background}

In 2020, some of us at UC Berkeley initiated a project to pivot more
campus courses towards mastery learning [cite]. A key component of
that approach is frequent low-stakes assessments with flexibility in
scheduling and the possibility of re-takes [cite], so our long term
goal was to emulate the very successful approach of operationally
streamlined [cite my draft] computer-based testing facilities (CBTFs)
as demonstrated at the University of Illinois at Urbana-Champaign
[cite Craig’s paper].

We began with initial internal seed funding of around $75,000 and
later a pair of more expansive mastery-learning grants that covered
not only CBTF operations but research and content development related
to mastery learning, each around $600,000.

Our initial CBTF pilot involved a single software engineering course
of around 200 students, using a pair of shared instructional labs as
our proto-CBTF.

This report summarizes our experience in bootstrapping this
effort. Other reports discuss why the UIUC approach to computer-based
testing differs from a generalized proctoring facility [cite] and our
experience scaling up from the prototype described here to our current
scenario: a 30-seat dedicated CBTF with a 100-seat CBTF opening in
Fall 2026, currently serving over 20 courses and over 5,500 enrolled
students. Much of what we describe shares common elements with Downey
et al.'s recent experience report doing something similar at
UC~Riverside~\cite{downey-cbtf}.


\subsection{Design assumptions and Initial Conditions}

We began with the following assumptions and design principles:

\begin{itemize}

\item Some of the
UC~Berkeley that would most benefit from mastery learning have
enrollments of 1,000 or more.  Like Downey et al.~\cite{downey-cbtf},
our goal was to support testing for entire courses,
not just students with
special proctoring needs.
This  goal leads to  different 
operating assumptions than for a special-purpose or a generalized proctoring
center~\cite{fox-cbtf-not-proctoring}.  

\item PL, content authoring.  Mostly simple q's (define), a few
  bespoke ones



\end{itemize}

Starting assumptions – stack, CTBF, cite Kelly paper


\subsection{Shared Spaces}

2 shared spaces - instructional labs centrally operated, in 2 diff bldgs

Rule of thumb: Try to avoid doing things that won't scale, or
processes that will have to be redesigned later in order to
scale. Design processes so that exceptions create O(1) work per
student or per instructor rather than O(n) work for CBTF
proctors/staff


