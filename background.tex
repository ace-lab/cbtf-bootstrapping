\section{Background}

In 2020, some of us at UC~Berkeley initiated a project to pivot more
campus courses towards mastery learning [cite]. A key component of
that approach is frequent low-stakes assessments with flexibility in
scheduling and the possibility of re-takes [cite], so our long term
goal was to emulate the very successful approach of operationally
streamlined [cite my draft] computer-based testing facilities (CBTFs)
as demonstrated at the University of Illinois at Urbana-Champaign
[cite Craig’s paper].

We began with initial internal seed funding of around \$75,000 and
later a pair of more expansive mastery-learning grants that covered
not only CBTF operations but research and content development related
to mastery learning, each around \$600,000.

Our initial CBTF pilot involved a single software engineering course
of around 200 students, using a pair of shared instructional labs as
our proto-CBTF.

This report summarizes our experience in bootstrapping this
effort. Other reports discuss why the UIUC approach to computer-based
testing differs from a generalized proctoring facility~\cite{fox-cbtf-not-proctoring}
and our experience scaling up from the prototype described here to our
current scenario: a 30-seat dedicated CBTF with a 100-seat CBTF
opening in Fall 2026, currently serving over 20 courses and over 5,500
enrolled students. Our experience has much in common with that
reported by  Downey et al.\ at UC~Riverside~\cite{downey-cbtf}.


\subsection{Design assumptions and Initial Conditions}

We began with the following assumptions and design principles:

\begin{itemize}

\item Some of the
UC~Berkeley that would most benefit from mastery learning have
enrollments of 1,000 or more.  Like Downey et al.~\cite{downey-cbtf},
our goal was to support testing for entire courses,
not just students with
special proctoring needs.
This  goal leads to 
scale-friendly operating assumptions that are very different from
those that would obtain for a special-purpose or a generalized proctoring
center~\cite{fox-cbtf-not-proctoring}.  

\item We found much to admire about the
  PrairieLearn\footnote{prairielearn.org} assessment authoring and
  delivery system and intended to use it.  In particular, although 
  Canvas is the official campuswide Learning Management System, we
  found its quiz abilities impoverished with respect to both the kinds
  of questions that can be authored and the operational facilities for
  delivering them in an environment designed to thwart cheating.  We
  discuss this a bit further in Section~\ref{sec:canvas}.

\item If the pilot succeeded, we planned to expand to other courses
  gradually, and eventually turn over CBTF operations to a central
  campus unit; at our institution, the most likely home would be
  Research, Teaching, and Learning (RTL), which among other
  responsibilities stewards our instructional computing facilities and
  oversees many campuswide software deployments (including Canvas).

\end{itemize}

\subsection{The Pilot Course}

\todo{I think we had more than 217 students but a few couldn't be
  accommodated in the CBTF.}
Our pilot course had five quizzes designed to take about an hour each
that were taken remotely.  Students were required to setup
self-proctoring in a such a way that they were both capturing their
activity as a screencast, and had a separate second camera (such as a
cell phone) recording video of the student taking the exam, with 
the student's face/hands visible. Students had to submit both the
screencast and the live video.

The final exam was administered to 217 students in our proto-CBTF over
a four-day period with two to three 3-hour sessions available for reservation each day,
for a total of ten sessions.

\subsection{Shared Spaces}

\begin{figure*}
  \caption{\label{fig:spaces}%
    Our two shared lab spaces are similarly configured with 27 (left)
    and 30 (right) desktop Windows PCs with Ethernet networking on
    long shared tables.
  }
  \includegraphics[width=\textwidth]{figs/labs}
\end{figure*}

As Figure~\ref{fig:spaces} shows, the two shared labs available to us
are similarly configured and designed more for lab work than exam
administration.
At the time, the computers were running Windows~10, and some were
several PC generations old.
The lab is fairly heavily scheduled so we had to book our time well in
advance for the midterm (2 hours) and final exam (3 hours) 


2 shared spaces - instructional labs centrally operated, in 2 diff bldgs

Rule of thumb: Try to avoid doing things that won't scale, or
processes that will have to be redesigned later in order to
scale. Design processes so that exceptions create O(1) work per
student or per instructor rather than O(n) work for CBTF
proctors/staff


