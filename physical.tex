\section{Physical Configuration}

\subsection{Privacy}

We purchased anti-glare/privacy screens for about \$30 per
workstation\footnote{\url{https://www.amazon.com/dp/B0C9L8NQLW}} that
can be affixed to monitors using Velcro, since we could not make
permanent changes to the shared computers.
When it was the CBTF's turn to use the labs, we would arrange for our
reservation to begin (and proctors to arrive) 30 minutes before the
first reservable exam shift to put the screens on,  then stay 30 min
late to take them off.
This implies the CBTF also needs a place where this and related
supplies can be stored.
In our case, the other supplies included a few pairs of
noise-cancelling headphones, and packs of scratch paper color-coded
for different sessions.


Get partitions for reduced distraction (or general privacy)


Have a way for students to leave personal belongings (incl. watches, phones, etc) clearly away from testing station. On ground is OK but harder to spot. Beware of in jacket pockets/baggy sweaters, etc.  Best is apply same rule to all: no personal devices near you, period. CAVEAT: 2FA.


\subsection{Accessibility}


XT should be “easy” to handle


Can any stations be setup for RD? Work with your campus DSP to do this, or else punt on it. Don’t assume you know what RD means


\subsection{Equipment and Facility Maintenance}

centrally administered, cleaned, HW maint, etc

well-posted method for notifying of HW/networking failures, including
while using

needed to coordinate for access - physical keys vs keycards.  proctors
were given access and/or knew the process for obtaining and returning
physical keys

