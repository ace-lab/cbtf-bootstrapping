\section{IT Involvement}

\textbf{Lesson:} Let PrairieLearn host for you.  
Understand and then minimize the interface with campus IT, to minimize the
overall number of moving parts.  There are a few aspects of operations
that probably only they can handle.

\subsection{Software: PrairieLearn}

We initially hosted PrairieLearn ourselves, since it was open source
and we didn't know if we'd need to make changes to integrate with our
campus systems.  In retrospect, this was a mistake.

Our IT staff is highly competent and were very cooperative.  However,
this represented a brand-new set of tools and
workflows for them.  Extra effort such as this must be charged to some account,
and even though we had research funds to pay for this, university IT staff time is typically tightly
budgeted.

We initially thought that our low usage would make it reasonable to
run locally: only one course actually trying to run CBTF-like exams,
during which smooth PrairieLearn operation is mission-critical,
and a few other courses using it for homework and practice, which are
typically not.  But as soon as more courses began doing CBTF quizzes,
the manual scaling of Amazon EC2 (Elastic Compute Cloud) resources to
ensure sufficient capacity quickly became a nuisance, with IT staff
needing a 24-hour heads up before any scale-up period.  PrairieLearn
hosting would have made this transparent.

In addition, the open-source software does not include PrairieTest for
scheduling the CBTF.  Even with only one course to schedule, we
underestimated how useful PrairieTest would be, as we would discover
when we tried to do reservations ourselves.


\subsection{Hardware}

PrairieTest exams require only a browser; any heavy lifting such as
code autograding is done on the server side by managing Docker
containers (another reason to have PrairieLearn host for you, although
as a caveat, a course requiring custom external grading costs more
than 2x per seat as one that doesn't).
The computers in the shared labs are several generations old and
were running Windows 10 at the time, and were fine for the task, since
PrairieLearn does a good job of adhering to widely-adopted Web
standards for the browser GUI and (importantly) accessibility.
When it came time to outfit a new 30-seat room as a full-time CBTF, we
purchased \$500 computers on Amazon.
More recently, in another shared lab we've used 15-year-old computers recently upgraded to
Windows 11, and they are still satisfactory.
In our shared labs, the computers were centrally managed, and each had
a set of ``login profiles'' that had different software installed.  We
arranged with IT to create a new profile for PrairieLearn that 
disables USB ports, to prevent students bringing USB-borne materials
to an exam, and has virtually no software installed other than a Web
browser and some audio files consisting of white noise for students
who want to block out audio distractions.

All of these computers have wired Ethernet; we know better than to
rely on wi-fi for something so mission-critical.

\subsection{Networking and 2FA}

More important is IP firewalling, either host-level or CIDR-level.
Firewalling is essential; don't try to use a lockdown browser or
similar scheme as it gets messy fast.
The PrairieLearn login profile has the
necessary firewall rules to block browser access to any web site other
than PrairieLearn and the sites required for students to login to our
campus SSO.

A major pitfall we did not expect involves 2-factor authentication
(2FA).  Our IT had contributed code to PrairieLearn to allow federated
SSO (single sign-on) login via CalNet, our Shibboleth-based
authentication system.  However, as a matter of campus policy, most
CalNet-accessible sites trigger a 2FA flow\footnote{\url{https://calnet.berkeley.edu/calnet-2-step}} that requires students to
use their cell phone to complete login.  This requirement led to
various failure modes:

\begin{itemize}

\item Wifi and/or cellular data coverage is weak in the basement where the labs are, so the phone
  cannot complete the 2FA flow.
\item The student recently obtained a new cell phone and has not set
  it up for 2FA.
\item The student normally uses a different method to complete 2FA,
  such as a physical USB token, tablet, or touchID-capable laptop,
  which may not be used in the CBTF.

\end{itemize}

The correct solution for us in our full-time CBTF is to disable 2FA
for logins on the CBTF's CIDR block.  But this is possible only with
close cooperation with campus IT (happily, the team responsible for
2FA and the team responsible for maintaining the shared instructional
labs are in the same campus unit) and because the full-time CBTF's
computers have their own CIDR block.

\subsection{Accessibility}

\textbf{Lesson:} There are more accommodations in heaven and earth
that are dreamt of in your philosophy.  Accessibility goes far beyond
the capabilities of the software, which in the case of PrairieLearn
are very good.

Once the pilot was over and we needed to start paying
PrairieLearn to host the software, as a vendor they needed to meet
strict standards of accessibility (a11y).  The good news is their software
does a great job of this, enough to satisfy our very finicky campus review.

However, there are aspects of a11y you must deal with during
a pilot, such as the physical a11y limitations of the room itself: Can
  wheelchairs be accommodated?  Standing desks?  Adjustable height
  desks?  
  
Try to get your Disabled Students Program (it has different names at
different institutions; we'll generically refer to it as DSP) on board
early.
We found this challenging
because initially they were quite defensive at the concept of a CBTF
serving an entire course, insisting that every seat be accessible, whereas
our goal was to identify which accommodations
can be met straightforwardly, and continue handling the rest by
one-off means.  \emph{Be sure you communicate to your DSP that while you're
hoping to accommodate a subset of students adequately within the
constraints of your CBTF, you will still accommodate all students as
needed, by having others take exams outside your CBTF.}

For example, some students have a ``reduced distraction during exams''
(RD) accommodation.
We were dismayed that we couldn't find any single set of rules
defining the physical parameters of this accommodation in a standard
way within UC Berkeley, let alone systemwide across the nine UC
campuses.
We had
seen photos of UIUC's CBTF in which a subset of RD testing stations
were created facing the wall and flanked by tall cubicle partitions
(to minimize visual distractions) and with noise-cancelling
headphones provided (to minimize auditory distractions).  In the end,
we were able to work with our DSP office to specify a configuration involving headphones and view
partitions that our DSP office agreed would satisfy the RD requirement
for most students, with the caveat that students who had more
stringent needs (e.g. ``must take the exam alone in a room'') would
still be accommodated, just not in the CBTF.

(Our DSP office has a proctoring facility for students needing
accommodations, but it is vastly oversubscribed.  We had been trying
from the start to convey the message that by handling ``common''
accomodations such as reduced distraction and extra time on exams, we
could relieve the DSP proctoring center of much workload.  It has
taken a long time to reach this understanding, but as of February 2026, over
85\% of all exam-related accommodations approved by DSP can be handled
within our CBTF.)

